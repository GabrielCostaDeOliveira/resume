\skills{Bacharelado em Engenharia de Software}, \\
\textit{Universidade de Brasília (UnB)}\strut \hfill 2019 --- 2025\\

\vspace*{7pt}

Como \textbf{Engenheiro} possuo formação robusta em matemática aplicada, tópicos esse que são fundamentais para inteligência artificial.

\vspace*{7pt}

\begin{tabular}{@{}ll@{}}
  \begin{minipage}[t]{0.45\textwidth}
    \begin{itemize}
      \item Cálculo 1, 2 e 3 (270h)
      \item Probabilidade e Estatística (60h)
      \item Matemática Discreta 1 e 2 (120h)
      \item Álgebra Linear (60h)
    \end{itemize}
  \end{minipage}
  &
  \begin{minipage}[t]{0.45\textwidth}
    \begin{itemize}
      \item Métodos Numéricos (60h)
      \item Processamento de Sinais (90h)
      \item Equações Diferenciais (60h)
      \item Inteligência Artificial (240h)
    \end{itemize}
  \end{minipage}
\end{tabular}

\vspace*{7pt}

%A Universidade de Brasília avançou no ranking CWUR de 2024, alcançando a 11ª posição entre as brasileiras,
%19ª na América Latina e 836ª no mundo, situando-se no top 4\% global. \href{https://cwur.org/2024/university-of-brasilia.php}{Veja mais}
