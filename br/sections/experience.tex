\documentclass[a4paper,12pt]{article}
\usepackage{TLCresume}
\begin{document}

%using zitemize instead of itemize will allow for tighter spacing before and after each line

%asdfsdfasdf asdfasdf


%====================
% EXPERIENCE A
%====================

\subsection*{\large AI.lab UnB \hfill Janeiro de 2024 --- Atualmente}
\subtext{\hfill Brasília, DF}

%\subtext{\normalsize Projeto Osiris}

\vspace{0.5em}

Atuo como \textbf{desenvolvedor backend} no projeto Osíris, uma iniciativa de \textbf{R\$ 3,5 milhões} financiada pelo Governo do Distrito Federal
para aprimorar a execução fiscal, em colaboração com a PGDF, o TJDFT e a UnB.
Sou responsável pela manutenção e testes de uma aplicação \textbf{backend} utilizada por \textbf{juízes e procuradores}.
Antes disso, trabalhei no mesmo projeto como \textbf{cientista de dados}, realizando processos de ETL e treinando modelos de NLP com ferramentas como Sklearn, PyTorch e LLMs (como o Llama 3.3).
\subtext{\href{https://www.pg.df.gov.br/projeto-osiris-ordem-de-servico-e-assinada-pelo-governador-do-df/}{Veja mais sobre o projeto}}

\vspace{1em}

% Função: Desenvolvedor Backend
\subsubsection*{\small \textit{Desenvolvedor Backend} }
\vspace{-1em}
\subtext{\tiny  Agosto de 2024 --- Atualmente}
\vspace{-0.5em}
\begin{itemize}
    \item Desenvolvimento de APIs utilizando \textbf{FastAPI} para suporte às soluções de automação de processos fiscais.
    \item Implementação de testes automatizados com \textbf{Pytest} para garantir a qualidade e robustez do código.
    \item Utilização de \textbf{Docker} para containerização e implantação eficiente dos serviços.
    \item Colaboração em um ambiente Linux, seguindo metodologias ágeis com Scrum.
    \item Controle de versão e colaboração com \textbf{Git} e \textbf{Gitlab}.
\end{itemize}

% Função: Cientista de Dados
\subsubsection*{\normalsize \textit{Cientista de Dados}}
\vspace{-1em}
\subtext{\tiny Janeiro de 2024 --- Agosto de 2024}
\vspace{-0.5em}
\begin{itemize}
    \item Utilização do \textbf{MLflow} para gerenciar o ciclo de vida dos modelos de machine learning, incluindo o rastreamento de experimentos, versionamento de modelos e implantação.
    \item Responsável pela mineração, limpeza e aumento de dados para \textbf{identificar \textit{features} relevantes }para modelos de classificação.
    \item Colaboração implementação e integração dos modelos de \textbf{IA} com foco em processamento de \textbf{linguagem natural} (NLP).
    \item Colaboração na criação, implementação e integração dos modelos de \textbf{IA generativa} com foco em processamento de \textbf{linguagem natural} (NLP).
    \item Utilização de modelos LLM como \textbf{Llama3} e \textbf{Mistral}, além de técnicas de 
    \textbf{Machine Learning} com\textbf{ scikit-learn} e \textbf{Deep Learning} com \textbf{PyTorch.}
    \item Utilização do \textbf{Pandas} e do formato \textbf{Parquet} para manipulação e análise de grandes conjuntos de dados.
\end{itemize}

%====================
% EXPERIENCE B
%====================
\vspace{1.5em}
\vspace*{7pt}
\vspace{1.5em}


\subsection*{\large Zeroo \hfill Setembro de 2022 --- Novembro de 2023}
\subtext{Desenvolvedor Android \hfill Brasília, DF}

\vspace{-0.5em}
\begin{itemize}
    \item Treinamento de um modelo de classificação de objetos utilizando \textbf{TensorFlow} e\textbf{ ML Kit}, e sua implementação no aplicativo Android, garantindo a integração eficaz com o sistema existente.
    \item Integração de modelos de inteligência artificial do \textbf{ML Kit}, incluindo funcionalidades de reconhecimento de imagem em tempo real, \textbf{OCR} (Reconhecimento Óptico de Caracteres) e \textbf{processamento de linguagem natural}.
    \item Desenvolvimento integral de um aplicativo Android nativo usando Java/Kotlin, abrangendo todas as etapas desde a modelagem de requisitos até a implementação final.
    \item Liderança na modelagem de requisitos e documentação completa do projeto utilizando Mkdocs.
    \item Planejamento e implementação da arquitetura do aplicativo seguindo o padrão MVVM (Model-View-ViewModel).
    \item Desenvolvimento e documentação do banco de dados relacional interno do aplicativo com \textbf{SQLite3}, utilizando Room e LiveData.
    \item Concepção e documentação de \textbf{endpoints RESTful} para a comunicação do aplicativo, incluindo solicitação de melhorias e garantia de implementação correta dos endpoints.
\end{itemize}


%==================== 
% EXPERIENCE C 
%==================== 

\vspace{1.5em}
\vspace*{7pt}
\vspace{1.5em}

\subsection*{\large Universidade de Brasília  \hfill Março de 2022 --- Agosto de 2022}
\subtext{MONITORIA EM ESTRUTURAS DE DADOS E ALGORITMOS \hfill Brasília, DF}
\vspace*{3pt}
Durante essa experiência, tive a oportunidade de consolidar meus conhecimentos em estrutura de dados e algoritmos, além de aprimorar minhas habilidades de comunicação para transmitir esse conhecimento. Abordamos uma ampla gama de temas, incluindo:

\begin{itemize}
    \item Árvores
    \item Grafos 
    \item Complexidade Computacional e notação Big-O
\end{itemize}


%====================
% EXPERIENCE X
%====================


%\subsection{{Title \hfill Start Month YEAR --- Present}}
%\subtext{Employer Name \hfill City, State}
%Summary text
%\vspace*{3pt}
%\begin{zitemize}
%\item Lorem ipsum dolor sit amet, consectetur adipiscing elit.
%\item Suspendisse pretium elit quis ullamcorper ultricies.
%\item Morbi a nisl sit amet massa ultricies fermentum.
%\end{zitemize}



\end{document}


