\documentclass[a4paper,12pt]{article}
\usepackage{TLCresume}
\begin{document}

%using zitemize instead of itemize will allow for tighter spacing before and after each line


%====================
% EXPERIENCE A
%====================

\subsection*{\large AI.lab UnB \hfill Janeiro de 2024 --- Atualmente}
\subtext{\hfill Brasília, DF}

%\subtext{\normalsize Projeto Osiris}

Atuo em um projeto de alto impacto, chamado Osíris, que surgiu como resultado do \textbf{investimento de R\$ 3,5
milhões} do governo do Distrito Federal, onde desenvolvo uma aplicação que utiliza IA
para agilizar o trabalho de \textbf{juízes e procuradores} no contexto da execução fiscal. Aqui, tive a oportunidade
de atuar em duas posições diferentes, como cientista de dados e como desenvolvedor backend.

\vspace{0.5em}

\vspace{1em}

\subsubsection*{\small \textit{Desenvolvedor Backend} }
\vspace{-1em}
\subtext{\tiny  Agosto de 2024 --- Atualmente}
\vspace{-0.5em}
\begin{zitemize}
    \item Desenvolvi APIs e serviços de IA para integração de modelos de inteligência artificial para aprimorar o trabalho de juízes e procuradores na execução fiscal.
    \item deploy de modelos em produção, utilizando  \textbf{FastAPI}, contêineres (Docker) e \textbf{MLOps}
    \item Implementei testes unitários e de integração com \textbf{Pytest}.
    \item Utilização de técnicas de programação paralela e distribuída para reduzir o tempo de processamanto.
    \item Criação de novos módulos seguindo a arquitetura de \textbf{microsserviços}.
\end{zitemize}

\subsubsection*{\normalsize \textit{Cientista de Dados}}
\vspace{-1em}
\subtext{\tiny Janeiro de 2024 --- Agosto de 2024}
\vspace{-0.5em}
\begin{zitemize}
    \item Usei o \textbf{MLflow} para versionamento e desenvolvimento de \textbf{pipelines} de modelos de inteligência artificial,
    \item Usei o \textbf{MLflow} para análises de performance do modelo e testes de diferentes hipóteses,
    \item Implementei pipelines eficientes para pré-processamento (mineração, limpeza e aumento de dados), vetorização e armazenamento de dados textuais,
    \item Uso de LLM como \textbf{llama3}, além de técnicas de \textbf{RAG} e \textbf{NLP/PLN} para classificação de textos jurídicos.
    \item treinei e integrei com sucesso modelos de \textbf{inteligência artificial} para classificação de textos juridicos.
    \item \textbf{Machine Learning} com \textbf{ scikit-learn} e \textbf{Deep Learning} com \textbf{PyTorch} e  \textbf{TensorFlow}
    \item \textbf{Pandas} e \textbf{Parquet}.
\end{zitemize}

%====================
% EXPERIENCE B
%====================
\vspace{1.5em}
\vspace*{7pt}
\vspace{1.5em}

\subsection*{\large Zeroo \hfill Setembro de 2022 --- Novembro de 2023}
\subtext{Desenvolvedor Android \hfill Brasília, DF}

\vspace{-0.5em}
\begin{zitemize}
    \item Desenvolvimento de um aplicativo Android nativo usando Java/Kotlin
    \item Treinamento e integração no aplicativo de modelos de \textbf{inteligência artificial} para classificação de objetos utilizando \textbf{TensorFlow} e \textbf{ML Kit}.  
    \item Modelagem de requisitos e documentação do projeto utilizando Mkdocs.
    \item Utilização de arquitetura de software como MVVM (Model-View-ViewModel) e SOLID e \textbf{padroes} GOF
    \item Desenvolvimento e documentação do \textbf{banco de dados relacional} interno do aplicativo com \textbf{SQLite3}, utilizando Room e LiveData.
\end{zitemize}


%==================== 
% EXPERIENCE C 
%==================== 

\vspace{1.5em}
\vspace*{7pt}
\vspace{1.5em}

\subsection*{\large Universidade de Brasília  \hfill Março de 2022 --- Agosto de 2022}
\subtext{MONITORIA EM ESTRUTURAS DE DADOS E ALGORITMOS \hfill Brasília, DF}
\vspace*{3pt}

Durante essa experiência, tive a oportunidade de consolidar meus conhecimentos em estrutura de dados e algoritmos, além de aprimorar minhas habilidades de explicar conceitos complexos.

\begin{zitemize}
    \item Algoritmos gulosos
    \item Programação Dinâmica
    \item Árvores
    \item Grafos 
    \item Análise de complexidade (Big-O)
\end{zitemize}


%====================
% EXPERIENCE X
%====================


%\subsection{{Title \hfill Start Month YEAR --- Present}}
%\subtext{Employer Name \hfill City, State}
%Summary text
%\vspace*{3pt}
%\begin{zitemize}
%\item Lorem ipsum dolor sit amet, consectetur adipiscing elit.
%\item Suspendisse pretium elit quis ullamcorper ultricies.
%\item Morbi a nisl sit amet massa ultricies fermentum.
%\end{zitemize}



\end{document}
